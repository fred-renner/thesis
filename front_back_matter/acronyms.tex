%*******************************************************
% Acronyms
%*******************************************************

\chapter{Acronyms}
\begin{acronym}[neos]
    \acro{cern}[CERN]{Organisation européenne pour la recherche nucléaire}
    \acro{atlas}[ATLAS]{A Toroidal LHC Apparatus}

    % Theory
    \acro{sm}[SM]{Standard Model}
    \acro{qft}[QFT]{Quantum Field Theory}
    \acro{qcd}[QCD]{Quantum Chromodynamics}
    \acro{qed}[QED]{Quantum Electrodynamics}
    \acro{ew}[EW]{Electroweak}
    \acro{ewsb}[EWSB]{Electroweak Symmetry Breaking}
    \acro{vev}[VEV]{Vacuum Expectation Value}
    \acro{ckm}[CKM]{Cabibbo-Kobayashi-Maskawa}

    % \acro{ip}[IP]{impact parameter of tracks  in jets}
    % Detector
    \acro{lhc}[LHC]{Large Hadron Collider}
    \acro{hllhc}[HL-LHC]{High Luminosity \acs{lhc}}
    \acro{id}[ID]{Inner Detector}
    \acro{sct}[SCT]{semiconductor tracker}
    \acro{trt}[TRT]{transition radiation tracker}
    % \acro{itk}[ITk]{Inner Tracker}
    \acro{ibl}[IBL]{insertable $b$-layer}
    % \acro{em}[EM]{electromagnetic}
    % \acro{lar}[LAr]{liquid argon}
    % \acro{ms}[MS]{muon spectrometer}
    % \acro{rpc}[RPCs]{resistive plate chambers}
    % \acro{tgc}[TGCs]{thin gap chambers}
    % \acro{mdt}[MDTs]{monitored drift tubes}
    % \acrodefplural{mdt}[MDT]{monitored drift tubes}
    % \acro{csc}[CSCs]{cathod strip chambers}
    % \acrodefplural{csc}[CSCs]{Cathod strip chambers}
    \acro{hlt}[HLT]{high level trigger}
    % \acro{roi}[RoI]{region of interest}
    % \acrodefplural{roi}[RoIs]{regions of interest}
    \acro{l1}[L1]{Level-1}
    \acro{pfo}[PFO]{Particle Flow Object}
    \acro{tcc}[TCC]{Track CaloCluster}
    \acro{ufo}[UFO]{Unified Flow Object}

    % hh4b analysis
    \acro{ggf}[ggf]{gluon-gluon fusion}
    \acro{vbf}[vbf]{vector-boson fusion}
    \acro{nnlo}[NNLO]{next-to-next-to-leading order}
    \acro{nnnlo}[N$^3$LO]{next-to-next-to-next-to-leading order}

    
    % \acro{pdf}[PDF]{Parton Distribution Function}
    % \acro{dglap}[DGLAP]{Dokshitzer–Gribov–Lipatov–Altarelli–Parisi}
    % \acro{mc}[MC]{Monte Carlo}
    % \acro{mpi}[MPI]{multi-parton interaction}
    % \acro{ps}[PS]{parton shower}
    % \acro{me}[ME]{matrix element}
    % \acro{isr}[ISR]{initial state radiation}
    % \acro{fsr}[FSR]{final state radiation}
    % \acro{4fs}[4FS]{four-flavour scheme}
    % \acro{5fs}[5FS]{five-flavour scheme}
    % \acro{nlo}[NLO]{next-to-leading order}
    % \acro{}[]{}

    \acroplural{pdf}[PDF]{Probability Density Function}


    \acro{pv}[PV]{primary vertex}
    % \acro{jvt}[JVT]{jet vertex tagger}

    % \acro{ml}[ML]{Machine Learning}
    % \acro{mle}[MLE]{Maximum Likelihood Estimation}
    % \acro{llr}[LLR]{Log-likelihood ratio}
    % \acro{bdt}[BDT]{Boosted Decision Tree}
    % \acro{nn}[NN]{Neural Network}
    % \acro{relu}[\textsc{ReLU}]{Rectified Linear Unit}
    % \acro{adaboost}[AdaBoost]{Adaptive Boost}
    % \acro{HP}{Hyperparameter}


    % btagging
    % \acro{dl1}[DL1]{Deep Learning based heavy-flavour tagger}
    % \acro{wp}[WP]{working point}
    % \acro{vr}[VR]{variable radius}
    % \acro{ip}[IP]{impact parameter}
    % \acro{sv}[SV]{secondary vertex}
    % \acro{sv1}[SV1]{inclusive displaced secondary vertex reconstruction algorithm}
    % \acro{jf}[JF]{\textsc{JetFitter}}
    % \acro{smt}[SMT]{Soft Muon Tagger}
    % \acro{dips}[DIPS]{Deep Impact Parameter Sets}


    % \acro{sr}[SR]{signal region}
    % \acro{cr}[CR]{control region}
    % \acro{stxs}[STXS]{Simplified Template Cross-Section}

    % \acro{wlcg}[WLCG]{Worldwide LHC Computing Grid}
\end{acronym}

% more package info: https://www.namsu.de/Extra/pakete/Acronym.html
% Befehl	Wirkung
% ac{Kuerzel}	Bei der ersten Verwendung von ac{Kuerzel} wird die Langfassung der Abkürzung und die Abkürzung selbst in Klammern dargestellt. Wird der Befehl ac{Kuerzel} das nächste mal aufgerufen erschneit nur nocht die Abkürzung.
% \acresetall	Der Befehl \acresetall ermöglicht es das Gedächnis des ac Befehls zu löschen. Wird der Befehl \acresetall gesetzt verhält sich der ac Befehl danach wie beim ersten Aufruf (Bei allen bisher gesetzten Abkürzungen).
% acf{Kuerzel}	Mit acf{Kuerzel} gibt es ein zweites Erstes Mal für diese Abkürzung. Das heißt, sie wird wieder in der Langform und der geklammerten Abkürzung gezeigt.
% acs{Kuerzel}	acs{Kuerzel} gibt nur die Abkürzung aus.
% acl{Kuerzel}	acl{Kuerzel} gibt nur die Langform der Abkürzung aus.
% acp{Kuerzel}	Gleiche Wirkung wie ac{Kuerzel} nur hier wird der Plural ausgegeben.
% acfp{Kuerzel}	Gleiche Wirkung wie acf{Kuerzel} nur hier wird der Plural ausgegeben.
% acsp{Kuerzel}	Gleiche Wirkung wie acs{Kuerzel} nur hier wird der Plural ausgegeben.
% aclp{Kuerzel}	Gleiche Wirkung wie acl{Kuerzel} nur hier wird der Plural ausgegeben.
% acfi{Kuerzel}	Die Langform wird kursiv geschrieben, während die Abkürzung mit Kapitälchen dargestellt wird.
% iac{Kuerzel}	Hier wird der Abkürzung (beziehungsweise wenn es das erste Mal ist der Langform mit geklammerter Abkürzung) der unbestimmte englische Artikel a voran gestellt.
% Iac{Kuerzel}	Hier wird der Abkürzung (beziehungsweise wenn es das erste Mal ist der Langform mit geklammerter Abkürzung) der unbestimmte englische Artikel A voran gestellt.
% acused{Kuerzel}	Die Abkürzung wird als gesetzt markiert (gleiche Wirkung wie der ac Befehl) aber nicht angezeigt. Danach zeigt der ac Befehl nur noch die Abkürzung an.
% acsu{Kuerzel}	Zeigt die Abkürzung an und markiert sie als gesetzt.
% aclu{Kuerzel}	Zeigt die Langform an und markiert sie als gesetzt.