\begin{center}
\textbf{Abstract}
\end{center}
\noindent This thesis presents the first application of a novel optimization technique, termed - \acf{neos} -, in a search for Higgs boson pairs decaying into a boosted vector boson fusion topology consisting of four $b$-quarks in the final state at the ATLAS experiment. Particle physics research often involves complex, multi-stage analyses requiring iterative optimizations, typically reliant on signal-to-background ratios. While these conventional proxy optimizations are linked to  significance measures such as p-values, they do not take into account the significant role of systematic uncertainties when determining p-values, nor do they consider the complex interdependencies across the analysis pipeline. The \ac{neos} method revolutionizes this by simultaneously optimizing the p-value across all aspects of the analysis.

This research demonstrates that the \ac{neos} approach can effectively optimize any parameterizable method or parameter in a particle physics search, exemplified in this work by the optimization of event filtering and background estimation methods. It also introduces the concept of 'autoanalysis', where the algorithm autonomously determines the optimal analysis configuration for a given analysis strategy using only basic reconstructed physics objects. This innovation allows researchers to focus on developing analysis strategies rather than on manual optimization tasks.

The analysis uses the full Run 2 data from the ATLAS detector, incorporating the latest advancements in object reconstruction, such as Unified Flow Object large-radius jets and the GN2X version of the $X\rightarrow bb$-tagging algorithm. The \ac{neos} method has shown a two-fold improvement in setting expected cross-section limits over traditional methods that use the invariant mass of the Higgs pair system and a classifier neural network optimized for signal-to-background ratio.  With \ac{neos} and these reconstruction advancements, the \ktwov coupling parameter is constrained to \red{$(0.55 < \kappa_{2v}< 1.48)$} at the 95\% confidence level. This work establishes \ac{neos} as a robust tool for optimizing particle physics analyses, highlighting its potential for future research and discovery.

Additionally, this thesis presents an improved strategy for identifying $b$-quarks in small-$R$ jets, crucial for detecting Higgs pairs in the four $b$-quark final state. This method, which uses muons from semileptonic hadron decays, improves background rejection by 40%.

