The Standard Model of Particle Physics \ac{sm} is the current theory that describes three of the four fundamental forces, namely the electromagnetic, strong, and weak forces, with the exception of gravity. Over the last decades it has been probed with remarkable precision but although there are  observational phenomena that lie beyond its scope. 

The SM is based on symmetry principles and is described by a lorentz-invariant quantum field theory \ac{qft}. This qft is renormalizable and invariant under local gauge transformations belonging to the non-abelian gauge group 
\begin{equation}
    G = SU(3)_C \otimes SU(2)_L \otimes U(1)_Y,
\end{equation}
that leaves the equations of motions invariant under transformations within this group. $SU(3)_C$ is the special unitary group of rank 3 representing the color symmetry within Quantum Chromodynamics \ac{qcd}, the qft describing the strong interactions. $SU(2)_L \otimes U(1)_Y$ exhibits the unification of the weak and electromagnetic interaction into the electro-weak force of $SU(2)_L$ left-chiral fermions of the weak force and right-handed $U(1)_Y$ fermions with hypercharge $Y$ of the electromagnetic force. The following describes the particle content of the \ac{sm} and gives a brief overview of the qft's used to describe aforementioned forces. The content of this chapter draws inspiration primarily from \citep{hollik2010quantum,griffiths2020introduction,thomson2013modern}.


\section{Particle Content of the SM}


\section{Quantum Electrodynamics}