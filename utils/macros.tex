

\newcommand{\sg}[1]{\textcolor{blue}{#1}}
\newcommand{\ui}[1]{\textcolor{Green}{#1}}

\DeclareMathOperator{\Tr}{Tr}
\newcommand{\red}[1]{\textcolor{red}{#1}}
\newcommand{\wert}[3]{\SI[separate-uncertainty=true]{#1(#2)}{#3}}
\newcommand{\abs}[1]{\ensuremath{\left\vert#1\right\vert}}
\newcommand{\ket}[1]{\ensuremath{\left\vert #1\right>}}
\newcommand{\bra}[1]{\ensuremath{\left<#1\right\vert}}
\newcommand{\kasten}[1]{\mbox{\color{#1}$\blacksquare$}}
\newcommand{\rgbbox}[1]{\mbox{\color[RGB]{#1}$\blacksquare$}}
\newcommand{\hexbox}[1]{\mbox{\color[HTML]{#1}$\blacksquare$}}
\newcommand{\hexline}[1]{\mbox{\color[HTML]{#1}$\bm{\diagup}$}}
\newcommand{\dint}[1]{\ensuremath{\mathop{\mathrm{d}#1}}}
\newcommand{\vect}[1]{\mathbf{#1}}
\newcommand*{\pt}{\ensuremath{p_{\text{T}}}\xspace}
\newcommand*{\ktwov}{\ensuremath{\kappa_{\text{2V}}}\xspace}
\newcommand*{\cls}{\ensuremath{\mathrm{CL}_s}\xspace}
\newcommand{\ptrel}{$p_\mathrm{T}^\mathrm{rel}$\xspace}
\newcommand*{\ttbar}{\ensuremath{t\overline{t}}\xspace}






% \newcommand{\semileptonicDecay}{
%   \begin{tikzpicture}
%     \begin{feynman}
%       \vertex (a5) {\(b\)};
%       \vertex[right=1.5cm of a5] (a7);
%       \vertex[right=3.5cm of a5] (a6) {\(u\)};

%       \vertex[below=1.5em of a5] (b5) {\(d\)};
%       \vertex[right=3.5cm of b5] (b4) {\(d\)};

%       \vertex[above=of a6] (c1) {\(\nu_{\mu}\)};
%       \vertex[above=2em of c1] (c3) {\(\mu^{-}\)};
%       \vertex at ($(c1)!0.5!(c3) - (1cm, 0)$) (c2);

%       \diagram* {
%       {[edges=fermion]
%           (a5) -- (a7)-- (a6),(b4) -- (b5),
%         },
%       (c1) -- [fermion, out=180, in=-45] (c2) -- [fermion, out=45, in=180] (c3),
%       (a7) -- [boson, bend left, edge label=\(W^{-}\)] (c2),
%       };

%       \draw [decoration={brace}, decorate] (b5.south west) -- (a5.north west)
%       node [pos=0.5, left] {\( B^{0}\)};

%       \draw [decoration={brace}, decorate] (a6.north east) -- (b4.south east)
%       node [pos=0.5, right] {\(\pi^{+}\)};
%     \end{feynman}
%   \end{tikzpicture}
% }
 