\chapter{The HH$\rightarrow$4b analysis}\label{ch:hh4b}

Investigating the exact shape of the Higgs potential is an interesting endeavor, as it is directly related to \ac{ewsb} and fundamental questions about the nature of the universe, as discussed in section \ref{sec:beyond_sm}. The main Higgs production modes at the \ac{lhc} are shown in Figure \ref{fig:main_production_processes} and can be understood by the fact that the Higgs boson interacts with fermions via Yukawa couplings from equation \ref{eq:yukawa_term}. Since Yukawa couplings are directly proportional to the fermion masses the Higgs boson predominantly couples to heavier particles like the top quark or the massive vector bosons. All couplings are scaled relative to their \ac{sm} values and are denoted as $\kappa_\mathrm{c} = c/c_\mathrm{sm}$. A $\kappa_\mathrm{c}$ value of 1 therforee corresponds to the \ac{sm} value for some given coupling $c$.

The first two \ac{ggf} diagrams \ref{fig:main_production_processes}(a) and \ref{fig:main_production_processes}(b) have a cross-section of $\sigma_\text{vbf HH}^\text{SM}=\qty[]{31.05}{fb}$ calculated at a center of mass energy of \qty[]{13}{TeV} at \ac{nnlo} \citep{Grazzini_2018} while the \ac{vbf} processes (c), (d) and (e) of figure \ref{fig:main_production_processes} have a production cross-section of
$\sigma_\text{vbf HH}^\text{SM}=\qty[]{1.73}{fb}$ at \ac{nnnlo} \citep{PhysRevD.98.114016}. A characteristic of the \ac{vbf} processes is that the Higgs pair products are accompanied by two additional quarks. The \ac{vbf} cross section is about \qty[]{3e4}{} times smaller than the production cross section for single Higgs $\sigma_\text{H}^\text{SM}=\qty[]{48.58}{pb}$ at the \ac{lhc} \citep{de2016arxiv} and underlines the challenge of discovering Higgs pairs in these final states.
\begin{figure}
    \centering
    \subfigure[]{\includegraphics[width=.43\textwidth]{fig_01a}}\hspace{.06\textwidth}
    \subfigure[]{\includegraphics[width=.43\textwidth]{fig_01b}} \\
    \subfigure[]{\includegraphics[width=.3\textwidth]{fig_02a}}\hspace{.01\textwidth}
    \subfigure[]{\includegraphics[width=.3\textwidth]{fig_02b}}\hspace{.01\textwidth}
    \subfigure[]{\includegraphics[width=.3\textwidth]{fig_02c}}
    \caption[]{Leading Higgs Pair production processes at the \ac{lhc}. (a), (b) shows \ac{ggf} and (c), (d), (e) \ac{vbf} processes. Adopted from \citep{aad2023search}.}
    \label{fig:main_production_processes}
\end{figure}
% higgs hat keine Ladung whatsoever cannot couple em or qcd 

\begin{figure}
    \centering
    \includegraphics[width=0.7\textwidth]{branching_fraction_hh}
    \caption[]{Contributions of final states represented by area for a pair of Higgs. Adopted from \citep{ATL-COM-PHYS-2020-083}.}
    \label{fig:branching_fraction_hh}
\end{figure}
Figure \ref{fig:branching_fraction_hh} highlights that an interesting channel in the study of Higgs pair production is the final state with the largest branching fraction, which consists of four $b$ quarks and amounts to about \qty[]{34}{\percent}. Thus the \ac{sm} \ac{vbf} cross-section is calculated to correspond to the 4b branching ratio by multiplying it with $\mathcal{B}(4b)=0.3392$. This fully hadronic final state, however, presents the challenge of significant \ac{qcd} backgrounds.

This work focuses on the boosted topology of highly energetic jets which do not allow reconstruction of $b$-jets individually but rather of final states consisting of large-$R$ jets encapsulating two collimated $b$-jets inside. This approach substantially reduces \ac{qcd} backgrounds, as highly energetic jets are more likely to originate from heavy particles like $b$ quarks. Additionally, events with jets of large \pt are easier to trigger on. While representing a comparatively clean signal, such events are rare and thus have limited statistical power. Despite other decay signatures being more suitable for the discovery of the Higgs pair production process the power of this selection lies in proving the existence of the \ktwov coupling shown in figure \ref{fig:main_production_processes}(d) to which it is directly sensitive.

The low cross-section for this process is due to the fact that diagrams (d) and (e) in Figure \ref{fig:main_production_processes} exhibit destructive interference for \ac{sm} values. Conversely, when $\ktwov$ deviates from \ac{sm} values, the production cross-section increases significantly, with $\sigma_{\ktwov=0}\approx 20\sigma_{\ktwov=1}$ and the decay products exhibit much larger transverse momentum, as illustrated in Figure \ref{fig:kappa_2v_variations_mhh}.
\begin{figure}
    \centering
    \includegraphics[width=1\textwidth]{kappa_2v_variations_mhh}
    \caption[]{Invariant mass of the Higgs pair system and the leading Higgs candidate jet \pt  reconstructed from simulation for different \ktwov. Adopted from \citep{ATL-PHYS-PUB-2019-007}.}
    \label{fig:kappa_2v_variations_mhh}
\end{figure}

\section{Data and Monte Carlo Simulation}\label{sec:mc_simulation}
This analysis uses the full run 2 data taken by \ac{atlas} between 2015 and 2018. The dataset contains \qty[]{140.1}{fb^{-1}} of data good for physics at a center of mass energy of \qty[]{13}{TeV} \citep{DAPR-2021-01}.

\ac{mc} generation in \ac{atlas} is typically done in three steps. At first at parton level the matrix element of the process of interest is stochastically simulated with \textsc{MadGraph} (v.2.7.3p3.atlas6) \citep{alwall2014automated}. The cross sectional calculation for proton-proton collisions relies on the factorization theorem \citep{halzen1984introductory} which states that contributions from partons participating in the hard scatter event can be factorized. Further partons cannot be observed individually since the approximation of the perturbation ansatz of section \ref{sec:qft} breaks down for low energy scales $\mu^2$ as described in section \ref{sec:renormalization}. This is the energy scale for which the approximation would need to hold to describe the partons inside a proton. However parton densities can be studied within \ac{qcd} using the DGLAP equations \citep{thomson2013modern}. Similar to renormalization, a scaling behavior can be derived from these equations that allows to derive an estimate of the \acp{pdf} by measuring it at some factorization scale $\mu_F^2$ in order to extrapolate it to another. Figure \ref{fig:pdf} exemplifies this for two energy scales from the \textsc{NNPDF3.0nlo} \ac{pdf} set used in this analysis.
\begin{figure}
    \centering
    \subfigure[]{\includegraphics[width=.47\textwidth]{NNPDF-10}}
    \subfigure[]{\includegraphics[width=.47\textwidth]{NNPDF-10000}}
    \caption[]{\textsc{NNPDF3.0nlo} parton distribution functions for two different factorization scales (a) $\mu_F^2=$\qty{10}{GeV}$^2$ and (b) $\mu_F^2=$\qty{10}{TeV}$^2$ against the momentum fraction $x$ of the particle. Adopted from \citep{PhysRevD.98.030001}.}
    \label{fig:pdf}
\end{figure}
Thus for hadrons $A,B$ containing partons $a,b$ and their respective \acp{pdf} $f_{a}^A$ and $f_{a}^B$, dependent on the parton's momentum fraction $x$ and factorization scale $\mu_F^2$, the cross-section of a process $A,B\rightarrow X$ reads
\begin{equation}
    \sigma_{A,B\rightarrow X} = \sum_{a,b} \int_0^1 \text{d}x_1\text{d}x_2 f_a^A(x_1,\mu_F^2) f_b^B(x_2,\mu_F^2) \hat{\sigma}_{a,b\rightarrow X}(\alpha_s(\mu_R^2),\mu_R^2).
\end{equation}
$\hat{\sigma}_{a,b\rightarrow X}$ is the perturbatively calculable part and therefore depends on the strong coupling $\alpha_S$ and renormalization scale $\mu_R^2$.

In a second step the parton shower evolution including hadronization and initial and final state radiation is simulated with \textsc{Pythia8} \citep{Sjostrand:2014zea}. Figure \ref{fig:parton_shower} illustrates this process.
\begin{figure}[]
    \centering
    \includegraphics[width=.75\textwidth]{parton-shower}
    \caption{Simulation of an evolution of a proton proton collision: The red circle at the center is the hard collision and the purple oval a secondary hard scatter event. Both are surrounded by a tree-like structure of \ac{qcd} bremsstrahlung interactions simulated with a parton shower. Light green represent hadrons whereas their subsequent decays are shown in dark green. Photons are depicted in yellow. Adopted from \citep{Hoche:2014rga}.
        \label{fig:parton_shower}}
\end{figure}

In a final step the detector response of simulated final state particles is simulated with \textsc{Geant}4 \citep{Agostinelli:2002hh}. It models the detector geometry, the particle's path through the magnetic fields and the particle interactions with the detector material, potentially producing new particles or decays. The output of this step are energy deposits in the various subdetectors of \ac{atlas}. Subsequently these are passed on to a process known as digitization which models the readout electronics. The result of this is raw data being no different from that read out in the actual experiment.


\section{Linear combination of samples}
\newcommand{\kl}{\kappa_\lambda}
\newcommand{\kt}{\kappa_t}
\newcommand{\kvv}{\kappa_\text{2V}}
\newcommand{\kv}{\kappa_\text{V}}
\newcommand{\mhh}{m_\text{HH}}

This analysis is interested in constraining the couplings $\kappa_\text{V},\kappa_\lambda,\kappa_\text{2V}$ associated to the \ac{vbf} processes shown in figure \ref{fig:main_production_processes}. As computing resources are limited and MC generation is unfortunately computationally expensive only a few hypothesis can be simulated. However by exploiting the properties of the differential cross-sectional calculation, samples of any hypothesized coupling value can be created through linear combination of samples \citep{ATLAS-CONF-2019-049}. It is illustrative to consider the two \ac{ggf} diagrams with the $\kl$ and $\kt$ couplings when calculating the differential cross-section
\begin{align}
    \label{eqn:ggf_hh_feynSum}
    \frac{d\sigma(\kl, \kt)}{d \mhh} & =
    | A(\kt, \kl) |^2 = | \kl \kt M_{\bigtriangleup}(\mhh) + \kt^2 M_{\Box}(\mhh) |^2                                                                                                                                          \\
                                     & = \kl^2 \kt^2 |M_{\bigtriangleup}(\mhh)|^2                                                                                                                                              \\ & \quad+
    \kl \kt^3 [ M_{\bigtriangleup}^*(\mhh) M_{\Box}(\mhh)                                                              +                                                           M_{\Box}^*(\mhh) M_{\bigtriangleup}(\mhh) ] \\ &\quad +
    \kt^4 |M_{\Box}|^2                                                                                                                                                                                                         \\
                                     & = \kl^2 \kt^2 a_1(\mhh) + \kl \kt^3 a_2(\mhh) + \kt^4 a_3(\mhh).
\end{align}
When setting $\kt$ to its \ac{sm} value 1 the equation reduces to
\begin{equation}
    \label{eqn:ggf_hh_feynSimple}
    \frac{d\sigma(\kl)}{d \mhh} = \kl^2 a_1(\mhh) + \kl a_2(\mhh) + a_3(\mhh).
\end{equation}
The parameters $a_i$ depend non-trivially on $\mhh$. However for three given hypotheses of $\kl$ a linear system of equations with variables $a_i$ can be solved and thus $\frac{d\sigma(\kl)}{d \mhh}(\kl)$ is a function of $\kl$ only.

In complete analogy the squared expansion of the cross-sectional formula involving the three \ac{vbf} couplings  $\kappa_\text{V},\kappa_\lambda,\kappa_\text{2V}$ samples can be combined to produce any hypotheses from 6 simulated hypotheses
\begin{flalign}
    \label{eqn:vbf_hh_6term_chosen}
    \frac{d\sigma}{d\mhh}(\kvv, \kl, \kv) \nonumber =                                                                                                                                                                                                                                                                           \\ \nonumber
    \left(\frac{68 \kappa_{2V}^{2}}{135} - 4 \kappa_{2V} \kappa_{V}^{2} + \frac{20 \kappa_{2V} \kappa_{V} \kappa_{\lambda}}{27} + \frac{772 \kappa_{V}^{4}}{135} - \frac{56 \kappa_{V}^{3} \kappa_{\lambda}}{27} + \frac{\kappa_{V}^{2} \kappa_{\lambda}^{2}}{9}\right) \times \frac{d\sigma}{d\mhh}{\left(1,1,1 \right)}       \\ \nonumber
    + \left(- \frac{4 \kappa_{2V}^{2}}{5} + 4 \kappa_{2V} \kappa_{V}^{2} - \frac{16 \kappa_{V}^{4}}{5}\right) \times \frac{d\sigma}{d\mhh}{\left(\frac{3}{2},1,1 \right)}                                                                                                                                                       \\ \nonumber
    + \left(\frac{11 \kappa_{2V}^{2}}{60} + \frac{\kappa_{2V} \kappa_{V}^{2}}{3} - \frac{19 \kappa_{2V} \kappa_{V} \kappa_{\lambda}}{24} - \frac{53 \kappa_{V}^{4}}{30} + \frac{13 \kappa_{V}^{3} \kappa_{\lambda}}{6} - \frac{\kappa_{V}^{2} \kappa_{\lambda}^{2}}{8}\right) \times \frac{d\sigma}{d\mhh}{\left(1,2,1 \right)} \\ \nonumber
    + \left(- \frac{11 \kappa_{2V}^{2}}{540} + \frac{11 \kappa_{2V} \kappa_{V} \kappa_{\lambda}}{216} + \frac{13 \kappa_{V}^{4}}{270} - \frac{5 \kappa_{V}^{3} \kappa_{\lambda}}{54} + \frac{\kappa_{V}^{2} \kappa_{\lambda}^{2}}{72}\right) \times \frac{d\sigma}{d\mhh}{\left(1,10,1 \right)}                                 \\ \nonumber
    + \left(\frac{88 \kappa_{2V}^{2}}{45} - \frac{16 \kappa_{2V} \kappa_{V}^{2}}{3} + \frac{4 \kappa_{2V} \kappa_{V} \kappa_{\lambda}}{9} + \frac{152 \kappa_{V}^{4}}{45} - \frac{4 \kappa_{V}^{3} \kappa_{\lambda}}{9}\right) \times \frac{d\sigma}{d\mhh}{\left(1,1,\frac{1}{2} \right)}                                      \\
    + \left(\frac{8 \kappa_{2V}^{2}}{45} - \frac{4 \kappa_{2V} \kappa_{V} \kappa_{\lambda}}{9} - \frac{8 \kappa_{V}^{4}}{45} + \frac{4 \kappa_{V}^{3} \kappa_{\lambda}}{9}\right) \times \frac{d\sigma}{d\mhh}{\left(1,-5,\frac{1}{2} \right)}.
\end{flalign}
A validation of the method can be found in \citep{ATL-COM-PHYS-2023-033}.


% https://www.overleaf.com/project/638e1930f926cd21d5264259
% linear combination
% Unfortunately, MC generation is computationally expensive and time-consuming. As such, only1443
% a handful of MC simulation samples for only a handful of coupling values are actually produced, and a1444
% sample combination technique is employed to model the signal hypothesis across the coupling parameter1445
% space
% https://trexfitter-docs.web.cern.ch/trexfitter-docs/model_building/expression/
% https://gitlab.cern.ch/hh4b/hh4b-boosted-vbf-limits/-/blob/main/create_workspaces/configs/k2V_parameterized_BDT_decorXbb.config?ref_type=heads#L285-331

\section{Analysis strategy}
This section describes the event selection and analysis strategy. A detailed description of reconstructed physical objects used is described in chapter \ref{ch:reco}.

\subsection{Trigger}
As outlined in section \ref{sec:tdaq} events need to be preselected. The \ac{hlt} applied in this analysis selects events with a large transverse energy $E_\text{T}$ large-$R$ jet. The definition slightly changed over the data taking years as can be seen in table \ref{tab:trigger}.
\begin{table}[htbp]
    \centering
    \caption{Trigger selections per data taking year and minimum requirements on transverse energy $E_\text{T}$ and mass $m$ on the large R jet. }
    \begin{tabular}{ccc}
        \hline
        Year & $E_\text{T}$ & $m$   \\ \hline
        2015 & $>360$       & 0     \\
        2016 & $>420$       & 0     \\
        2017 & $>420$       & $>35$ \\
        2018 & $>420$       & $>40$ \\ \hline
    \end{tabular}
    \label{tab:trigger}
\end{table}
Previous studies have shown that they become fully efficient at about $\pt>\qty[]{420}{GeV}$ \citep{ATL-COM-PHYS-2020-083,ATL-COM-PHYS-2023-033}.
% Figure \ref{fig:trigger_eff} depicts the efficiencies for the different triggers that become fully efficient at about $\pt>\qty[]{420}{GeV}$.
% \begin{figure}
%     \centering
%     \includegraphics[width=1\textwidth]{trigger_eff}
%     \caption[]{{TODO MYSELF}}
%     \label{fig:trigger_eff}
% \end{figure}

\subsection{Large Radius Jets}
To fully capture the boosted Higgs pair topology two large $R=1.0$ jets clustered with the Anti-$k_t$ algorithm from \acp{tcc} are used as described in section \ref{sec:jets}. These enclose the two boosted collimated $b$-jets in each of them to form the Higgs candidates. If there are several large-$R$ jets the two with the highest \pt are chosen. To be fully efficient on the trigger the leading large-$R$ jet is required to have $\pt>\qty[]{450}{GeV}$. For decay products to be inside a jet holds approximately $R\approx 2m/\pt$ with the mass $m$ and transverse momentum of the parent particle \citep{ATLAS-CONF-2020-022}. For a Higgs mass of \qty[]{125}{GeV} to be contained inside a large-$R$ jet the Higgs candidate therefore must have $\pt\gtrsim \qty[]{250}{GeV}$ and is thus chosen as the \pt requirement on the sub-leading Higgs candidate. If there are several large-$R$ jets the two leading \pt jets are selected. Additionally both Higgs candidates have a mass requirement $m>\qty[]{50}{GeV}$ to reduce \ac{qcd} background.
% As a minimum quality criterion large R jets are then required to have $200<p_{\text{T}}<3000$ GeV, $50<m<600$ GeV and $|\eta|<2$.
The $X\rightarrow bb$ tagger described in \ref{sec:xbb} is used to identify b-jets within the selected large-R jets. The top fraction $f_\text{top}$ is set to 0.25 and the \qty[]{60}{\percent} Higgs efficiency \ac{wp} is required. Studies with the more inclusive \qty[]{70}{\percent} \ac{wp} displayed slightly worse limit results \citep{ATL-COM-PHYS-2023-033}.

\subsection{Small Radius Jets}
Two small radius $R=0.4$ jets are required for the \ac{vbf} signature and are referred to as \ac{vbf} jets in the following. They are also reconstructed with the anti-$k_t$ algorithm and as \acp{pfo} as described in \ref{sec:jets}. The tight \ac{wp} for the \ac{jvt} and the LooseBad \ac{wp} for the event cleaning are applied both described in \ref{sec:calibration}. Small-$R$ jets $j$ are selected for $\pt>\qty[]{20}{GeV}$ and $|\eta|<4.5$ and are required to be outside of the Higgs candidate large-$R$ jets $J$ by imposing $\Delta R(J,j) > 1.4$. Further cuts applied on the \ac{vbf} jet system optimized on significance are $|\Delta\eta(j,j)| > 3$ and $m_{jj} > \qty{1}{TeV}$.

\subsection{Kinematic Regions}\label{sec:kinematic_regions}
\ac{sr}, Validation Region (VR) and \ac{cr} are explored and optimized in previous analyses \citep{aad2023search,ATL-COM-PHYS-2023-033} in the $m_{H1},m_{H2}$ plane and are defined as
\begin{equation}
    SR=X_{hh} =  \sqrt{\left(\frac{m_{H1} - \SI{124}{\GeV}}{1500 / m_{H1}}\right)^{2} + \left(\frac{m_{H2} - \SI{117}{\GeV}}{1900 / m_{H2}}\right)^{2}} < 1.6,
\end{equation}
\begin{equation}
    \label{VR_Xhh}
    VR =  \sqrt{\left(\frac{m_{H1} - \SI{124}{\GeV}}{0.1 \ln(m_{H1})}\right)^{2} + \left(\frac{m_{H2} - \SI{117}{\GeV}}{0.1 \ln(m_{H2})}\right)^{2}} < 100,
\end{equation}
and
\begin{equation}
    \label{CR_Xhh}
    CR = \sqrt{\left(\frac{m_{H1} - \SI{124}{\GeV}}{0.1 \ln(m_{H1})}\right)^{2} + \left(\frac{m_{H2} - \SI{117}{\GeV}}{0.1 \ln(m_{H2})}\right)^{2}} > 100  \ \& \ < 170.
\end{equation}
Figure \ref{fig:m_hh_plane} depicts the regions in the $m_{H1},m_{H2}$ plane on the \ac{sm} signal sample.

\begin{figure}
    \centering
    \includegraphics[width=.7\textwidth]{m_hh_plane}
    % \subfigure[]{\includegraphics[width=.49\textwidth]{massplane_SR_opening}}
    \caption[]{\red{REDO}}
    \label{fig:m_hh_plane}
\end{figure}

\subsection{Background Estimation}\label{sec:abcd}
Since the final state of this analysis is hadronic it remains a challenging task to estimate the contributions from the plethora of \ac{qcd} processes that contribute to backgrounds via  misidentification of light quarks as heavy $(b, t)$-quarks. Therefore the well established ABCD method is employed to derive a data-driven background estimate \citep{buttinger2018background,PhysRevD.103.035021}. It is based on the idea to use two independent variables e.g. $f$ and $g$ to define four orthogonal regions A, B, C and D as illustrated in figure \ref{fig:abcd} so that for some combination of the ratio of the event yields in the regions hold
\begin{equation}
    \frac{N_A}{N_B}=\frac{N_C}{N_D}.
\end{equation}
\begin{figure}
    \centering
    \includegraphics[width=.7\textwidth]{abcd}
    \caption[]{Illustration of four orthogonal regions A,B,C and D defined by two variables $f$ and $g$ in the horizontal and signal and background yields in the vertical dimension. Adopted from \citep{PhysRevD.103.035021}.}
    \label{fig:abcd}
\end{figure}
By rearranging the equation for the unknown $N_A$ an estimate for the background of the signal region can be derived from the other known quantities that lie in the regions dominated by the background. This approach relies on the assumption that the shape of the background in figure \ref{fig:abcd} does not vary greatly between C to D and A to B. Therefore the method must always be tested in a different region to determine its reliability.


In this analysis the two orthogonal variables are defined via the amount of $X\rightarrow bb$ Higgs tagged large-$R$ jets denoted as Xbb and the kinematic regions of the \ac{sr} and \ac{cr} defined in \ref{sec:kinematic_regions}.
\begin{table}[htbp]
    \centering
    \caption{Four orthogonal region definitions for the ABCD method}
    \begin{tabular}{|c|c|}
        \hline
        2 Xbb in CR & 2 Xbb in SR \\ \hline
        1 Xbb in CR & 1 Xbb in SR \\ \hline
    \end{tabular}
    \label{tab:abcd}
\end{table}
This gives the four orthogonal regions shown in table \ref{tab:abcd}. Hence, the background in the \ac{sr} is estimated with a weight extracted from the \ac{cr}
\begin{equation}
    N_\text{SR}^\text{2Xbb}=\frac{N_\text{CR}^\text{2Xbb}}{N_\text{CR}^\text{1Xbb}} N_\text{SR}^\text{1Xbb} = w_\text{CR} N_\text{SR}^\text{1Xbb}=  \red{0.0081} \times N_\text{SR}^\text{1Xbb}.
\end{equation}
The method is validated in the \ac{vr} within statistical uncertainties as displayed in figure \ref{fig:bkg-validation}. It is noted that the previous analysis also studied a binned transfer-factor without but did not see any improvement \citep{ATL-COM-PHYS-2023-033}.
\begin{figure}
    \centering
    \includegraphics[width=0.7\textwidth]{atos_cls_5_sys_NOSYS.VR_xbb_2_compareABCD}
    \caption[]{The background estimated in the \ac{vr} from data with 1 xbb tag, agrees with data in the \ac{vr} with 2 xbb tags within statistical uncertainties.}
    \label{fig:bkg-validation}
\end{figure}





\subsection{Event Classification}
After the selection of events, a deep feed-forward neural network is employed to construct the final histogram for the statistical test. This neural network's training utilizes a novel approach \textsc{neos}, which is thoroughly discussed in Chapter \ref{sec:neos} and optimizes on the $CL_s$ quantity detailed in section \ref{sec:statistics}. Inputs to the neural network include 20 features, which are the four vectors ($\pt,\eta,\phi,m$) of the Higgs boson pair system, the individual Higgs candidates, and the two \ac{vbf} jets.

The network's architecture features three fully connected layers, each comprising 100 nodes, and concludes with a singular output node. The hidden layers are followed by a rectified linear unit activation function whereas the output node employs a sigmoid activation for classification. The architecture is thus defined as [20,100,100,100,1], indicating the sequence of layers from input to output. Figure \ref{fig:nominal-hist} displays the nominal expected histogram for this analysis.


\begin{figure}
    \centering
    \includegraphics[width=0.7\textwidth]{atos_cls_5_sys_NOSYS.SR_xbb_2_nominal_hist.pdf}
    \caption[]{Expected histogram the data-driven background estimate and a signal hypothesis with $\kappa_\text{2V}=0$ and other couplings set to their \ac{sm} value. \red{add other nominal hypotheses}}
    \label{fig:nominal-hist}
\end{figure}
