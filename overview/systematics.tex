% The text outlines systematic uncertainties in a physics analysis, distinguishing between Monte Carlo (MC) simulation uncertainties and those from data-driven backgrounds. MC uncertainties, per CP group recommendations, impact only the signal and are propagated accordingly. Luminosity uncertainty, at 0.83%, has minimal analysis impact. Jet-related uncertainties involve energy scale, resolution, and mass, corrected through established in situ methods and additional flavor/topology considerations. The \Xbb tagging efficiency discrepancy between data and MC is calibrated using $Z(\rightarrow \bb)+$ jets events, with overall stable calibration but some fluctuations at the 70% working point.

% The impact of \Xbb uncertainties is significant, affecting all signal events' BDT responses by about +70%/-50%. By de-correlating uncertainties across transverse momentum bins, the analysis's sensitivity is improved. Small-R jet uncertainties are calculated using the JetUncertainties tool, considering various effects like dijet balance and single-particle uncertainties.

% Theoretical uncertainties are considered for cross-section calculations and parton shower modeling. The latter compares \PYTHIA8 and \HERWIGV7 generated samples, applying an interpolation strategy for resonant signals. Reweighting uncertainties from ggF to VBF variations are deemed negligible after validation against simulation.

% Background uncertainties include normalization and shape variations. The normalization factor is derived from control regions with an uncertainty of 12.3% for non-resonant and 14.4% for resonant analyses. Shape uncertainties are estimated from validation regions, applying smoothing techniques like the 2-bin method to mitigate statistical fluctuations. Overall, the text concludes with a systematic summary table indicating the influence of each uncertainty on normalization and shape for both signal and background.
\chapter{Systematics Uncertainties}
Any measurement needs to consider uncertainties in order to determine its validity. In this analysis they can be divided into systematic errors for the reconstructed objects, uncertainties from theoretical calculations, methodological errors and statistical uncertainties and are described in the following

\section{Jet Uncertainties}
Jets are calibrated using well known reference objects as described in section \ref{sec:calibration}. These corrections are themselves subject to uncertainties related to detector effects, modeling and statistics leading to corrections of the jet energy and are collectively referred to as \ac{jes} \citep{atlas2021jet,Aaboud:2019aa}. Since simulations of jets have a higher accuracy than observed jets the uncertainties of the simulated jets are broadened to be consistent with the jets observed in the data. These uncertainties are known as \ac{jer}. Furthermore large-$R$ jets are additionally corrected for their mass. The uncer tainties related to this procedure are called \ac{jmr} \citep{ATLAS-CONF-2020-022}.

\section{$X\rightarrow bb$ Tagger Uncertainties}
The \ac{nn} of the $X\rightarrow bb$ tagger was trained using simulations leading to potential discrepancies in selection efficiencies between observed data and simulation. Calibration is conducted with $Z(\rightarrow b\overline{b})+$~jets and $Z(\rightarrow b\overline{b})+\gamma$ applying the same methodology as in \citep{ATL-PHYS-PUB-2021-035}. However as of this analysis the $b$-tagging algorithm for the \ac{vr} track jets has been updated to the DL1d algorithm described in section \ref{sec:b_tagging}. The differences between \ac{mc} and data are measured in large-$R$ jet \pt and the extracted scale factors and their corresponding combined systematic and statistical uncertainties are shown in figure \ref{fig:xbb_sf}.
\begin{figure}
    \centering
    \subfigure[]{\includegraphics[width=.49\textwidth]{SF_Xbb50_internal_09March2023}}
    \subfigure[]{\includegraphics[width=.49\textwidth]{SF_Xbb60_internal_09March2023}}
    \caption[]{Derived scale factors in large-$R$ jet \pt for the \textbf{(a)} \qty[]{50}{\percent} and \textbf{(b)} \qty[]{60}{\percent} \ac{wp} from the calibration of the $X\rightarrow bb$ tagger.}
    \label{fig:xbb_sf}
\end{figure}

\section{Theory Uncertainties}
The cross-section calculation for some process initiated by a proton proton collision calculated at $n$-th order has a functional form \citep{unc_recipe}
\begin{equation}
    \sigma^{(n)} = PDF(x_1, \mu_F)  PDF(x_2, \mu_F) \hat{\sigma}^{(n)}(x_1,x_2,\mu_R),
    \label{eq:xs_unc_1}
\end{equation}
with the \acfp{pdf} carrying momentum fraction $x_{1,2}$ of the partons and the factorization scale $\mu_F$. This scale is named after the assumption that the cross-sections of the initial particle can be calculated by factorizing it in its parton contributions \citep{halzen1984introductory}. The term $\hat{\sigma}^{(n)}$ in equation \ref{eq:xs_unc_1} is the calculable part of the cross-section at renormalization scale $\mu_R$ as described in section \ref{sec:renormalization} and is expanded to a desired order $n$ in the strong coupling constant $\alpha_s$ with the usual \ac{qft} ansatz outlined in section \ref{sec:qft}
\begin{equation}
    \hat{\sigma}^{(n)} = \alpha_s \hat{\sigma}^{(0)} + \alpha_s^2 \hat{\sigma}^{(1)} + \ldots + \alpha_s^n \hat{\sigma}^{(n)} + \mathcal{O}(\alpha_s^{n+1}).
    \label{eq:xs_unc_2}
\end{equation}
Modelling cross-sections via \acp{pdf} is necessary since the approximation of the perturbation ansatz of section \ref{sec:qft} breaks down for low energy scales $Q^2$ as described in section \ref{sec:renormalization} which is the energy scale for which the approximation would need to hold to describe the partons inside a proton. However similar to renormalization a scaling behavior can be derived which allows to deduce an estimate of the \acp{pdf} by measuring it at a some energy scale $Q^2$ to extrapolate it to another. The equations enabling this are also expanded in $\alpha_s$ to a desired order and are knwon as DGLAP equations \citep{halzen1984introductory}. Three main sources of uncertainty arise in this calculation described in the following.

\subsubsection*{Scale Variations}
$\alpha_s$ is expanded to some order $n$ in the cross-section calculation and as well in estimating the \acp{pdf}. To account for missing higher orders corrections of theses expansions scale variations of the renormalization and factorization scales are performed pairwise $\{\mu_\text{r},\mu_\text{f}\}\ \times \{0.5,0.5\}, \{1,0.5\}, \{0.5,1\}, \{1,1\}, \{2,1\}, \{1,2\}, \{2,2\}$. For the cross-section calculation this accounts essentially for the term $\mathcal{O}(\alpha_s^{n+1})$ in equation \ref{eq:xs_unc_2}. The envelope that gives the largest variation is taken as the scale uncertainty.

\subsubsection*{\ac{pdf} Uncertainties}
\acp{pdf} need to be deduced from experiment and thus come by themselves with experimental uncertainties. Further uncertainties arise from the functional forms assumed for the \acp{pdf}.

\subsubsection*{$\alpha_s$ Uncertainties}
$\alpha_s$ is also experimentally deduced at the scale of the $Z$ mass which is subject to uncertainties. In all perturbative calculations it is truncated at some order that needs to be accounted for. 

The uncertainties on $\alpha_s$ and the \acp{pdf} are both estimated by varying $\alpha_s$. Even though there correlation is not strong they are usually applied combined \citep{unc_recipe}.

\subsection{Uncertainty on HH cross section}
The cross-section calculation for the \ac{vbf} Higgs pair production process has associated uncertainties for the scale variations $^{+0.03\%}_{-0.04\%}$ and the combined \ac{pdf}+$\alpha_s$ uncertainty is $\pm 2.1\%$ \citep{de2016arxiv}. 

\subsection{Uncertainty on Acceptance}
Theoretical uncertainties on the final acceptance are evaluated on \ac{mc} simulations for scale variations and \acp{pdf} + $\alpha_s$ \red{TODO, although shouldnt matter...} 

% pdf alpha_s easy TODO
\subsection{Parton Shower}
\textsc{Pythia 8} and \textsc{Herwig 7} are compared. \red{TODO}


\subsection{Branching Ratio Uncertainty}
The error estimate for the branching ratio takes into account theoretical uncertainties (THU) and parametric uncertainties (PU) that are included in the \ac{sm} calculations. The theoretical uncertainties mainly considers missing higher orders while for the parameters $p$ the four leading non-negligible contributions of $p=\alpha_s,m_c,m_b,m_t$ are considered.

Parametric uncertainties are Gaussian errors and are added in quadrature which ensures unity in the Branching Ratio calculation. Theoretical uncertainties in turn are not Gaussian and would lead to underestimated errors and are therefore are added linearly \citep{de2016arxiv}. By assuming a Higgs mas of \qty[]{125}{GeV} since and considering that there are two Higgs decaying to two $b$-quarks the error on the branching is 
\begin{equation}
\Delta\text{BR} = 2 \times \left(\Delta\text{BR}(\text{THU}) + \sqrt{\sum\nolimits_{p} \Delta\text{BR}(\text{PU}_{p})^2 }\right) = _{-3.5\%}^{+3.4\%}.
\end{equation}

\subsection{Reweighting Uncertainties}

\section{Background Modelling Uncertainties}

\section{Statistical Uncertainties}
As discussed in the chapter on statistics \ref{sec:statistics} the bin content for histograms in this work follows a Poisson distribution thus the standard error for $N$ events is the square root of the variance $\sigma=\sqrt{\text{Var}}=\sqrt{N}$. Since histograms are filled weighted this needs to be taken into account. The variance of a bin filled with weights $w_i$ is $\sigma^2 = \text{Var}_\text{bin}(w_i)$. Since the variance is additive and invariant with respect to constants it follows for the variance of the bin decomposed per event 
\begin{align}
    \sigma_\text{stat}^2 &= \text{Var}_\text{bin}\left(\sum_i w_i\right)
    =
    \underbrace{\sum_i \text{Var}(w_i \times 1\text{ event})}_{\text{Var}(i+j)=\text{Var}(i)+\text{Var}(j)}
    =
    \underbrace{\sum_i w_i^2\text{Var}(1\text{ event})}_{\text{Var}(aX)=a^2\text{Var}(X)}\\
    &=\sum_i w_i^2\sqrt{(1\text{ event})},
\end{align}
so that the statistical error of a bin is
\begin{equation}
    \sigma_\text{stat}^\text{bin}=\sqrt{\sum_i w_i^2}
\end{equation}

methodische errors, bkg unc


modelling errors


luminosity

