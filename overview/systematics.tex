\chapter{Systematic Uncertainties}
Any measurement needs to consider uncertainties in order to determine its validity. In this analysis, these uncertainties are classified into various types: systematic errors related to the reconstructed objects, uncertainties arising from theoretical calculations, methodological errors, and statistical uncertainties. Each of these categories is explained in detail in the subsequent sections.

\section{Luminosity}
The combined integrated luminosity for the years 2015-2018 has an uncertainty of \qty[]{0.83}{\percent} determined with the LUCID-2 detector. It is applied to the Higgs Pair signal process and has minimal impact on this analysis.

\section{Jet Uncertainties}
Jets are calibrated using well known reference objects as described in section \ref{sec:calibration}. These corrections are themselves subject to uncertainties related to detector effects, modeling and statistics leading to corrections of the jet energy and are collectively referred to as \ac{jes} \citep{atlas2021jet,Aaboud:2019aa}. Since simulations of jets have a higher accuracy than observed jets the uncertainties of the simulated jets are broadened to be consistent with the jets observed in the data. These uncertainties are known as \ac{jer}. Furthermore large-$R$ jets are additionally corrected for their mass similar to the approach for the \ac{jes}. The uncertainties related to this procedure are called \ac{jmr} \citep{ATLAS-CONF-2020-022}.

\section{$X\rightarrow bb$ Tagger Uncertainties}
The \ac{nn} of the $X\rightarrow bb$ tagger was trained using simulations leading to potential discrepancies in selection efficiencies between observed data and simulation. Calibration is conducted with $Z(\rightarrow b\overline{b})+\text{jets}$ and $Z(\rightarrow b\overline{b})+\gamma$ applying the same methodology as in \citep{ATL-PHYS-PUB-2021-035}. However as of this analysis the $b$-tagging algorithm for the \ac{vr} track jets has been updated to the DL1d algorithm described in section \ref{sec:b_tagging}. The differences between \ac{mc} and data are measured in large-$R$ jet \pt and the extracted scale factors and their corresponding combined systematic and statistical uncertainties are shown in figure \ref{fig:xbb_sf}.
\begin{figure}
    \centering
    \subfigure[]{\includegraphics[width=.49\textwidth]{SF_Xbb50_internal_09March2023}}
    \subfigure[]{\includegraphics[width=.49\textwidth]{SF_Xbb60_internal_09March2023}}
    \caption[]{Derived scale factors in large-$R$ jet \pt for the \textbf{(a)} \qty[]{50}{\percent} and \textbf{(b)} \qty[]{60}{\percent} \ac{wp} from the calibration of the $X\rightarrow bb$ tagger.}
    \label{fig:xbb_sf}
\end{figure}

\section{Theory Uncertainties}
The cross-section calculation for some process initiated by a proton proton collision calculated at $n$-th order as described in section \ref{sec:mc_simulation} has a functional form
\begin{equation}
    \sigma^{(n)} = PDF(x_1, \mu_F)  PDF(x_2, \mu_F) \hat{\sigma}^{(n)}(x_1,x_2,\mu_R),
    \label{eq:xs_unc_1}
\end{equation}
with the \acfp{pdf} carrying momentum fraction $x_{1,2}$ of the partons and the factorization scale $\mu_F$ \citep{unc_recipe}. The term $\hat{\sigma}^{(n)}$ in equation \ref{eq:xs_unc_1} is the calculable part of the cross-section at renormalization scale $\mu_R$ as described in section \ref{sec:renormalization} and is expanded to a desired order $n$ in the strong coupling constant $\alpha_s$ with the usual \ac{qft} ansatz outlined in section \ref{sec:qft}
\begin{equation}
    \hat{\sigma}^{(n)} = \alpha_s \hat{\sigma}^{(0)} + \alpha_s^2 \hat{\sigma}^{(1)} + \ldots + \alpha_s^n \hat{\sigma}^{(n)} + \mathcal{O}(\alpha_s^{n+1}).
    \label{eq:xs_unc_2}
\end{equation}
Similar to renormalization a scaling behavior can be derived which allows to deduce an estimate of the \acp{pdf} by measuring it at some energy scale $\mu_F^2$ to extrapolate it to another. The equations enabling this are also expanded in $\alpha_s$ to a desired order and are known as DGLAP equations \citep{halzen1984introductory}. Three main sources of uncertainty arise in this calculation described in the following.

\subsubsection*{Scale Variations}
$\alpha_s$ is expanded to some order $n$ in the cross-section calculation and as well in estimating the \acp{pdf}. To account for missing higher orders corrections of these expansions scale variations of the renormalization and factorization scales are performed pairwise $\{\mu_\text{r},\mu_\text{f}\}\ \times \{0.5,0.5\}, \{1,0.5\}, \{0.5,1\}, \{1,1\}, \{2,1\}, \{1,2\}, \{2,2\}$. For the cross-section calculation this accounts essentially for the term $\mathcal{O}(\alpha_s^{n+1})$ in equation \ref{eq:xs_unc_2}.

\subsubsection*{\ac{pdf} + $\alpha_s$ Uncertainties}
\acp{pdf} need to be deduced from experiment and thus have experimental uncertainties. Further uncertainties arise from the functional forms assumed for the \acp{pdf}. $\alpha_s$ is also experimentally deduced at the scale of the $Z$ mass and thus subject to uncertainties. In all perturbative calculations it is truncated at some order that needs to be accounted for \citep{unc_recipe,Butterworth_2016}.

Theoretical uncertainties on the final acceptance are evaluated on \ac{mc} simulations for scale variations and for \acp{pdf} and $\alpha_s$. For the scale variation the envelope the envelope encompassing all scale variations is used as the uncertainty.

The uncertainty introduced by the \acp{pdf} is calculated as a standard deviation of the cross section from the nominal \ac{pdf} set $\sigma^{(0)}$ \citep{Butterworth_2016}
\begin{equation}
    \Delta^\text{PDF}\sigma = \sqrt{\sum_k \left(\sigma^{(k)} - \sigma^{(0)}\right)^2}.
\end{equation}

The uncertainty on $\alpha_s=0.1180\pm0.0015$ is estimated with the associated uncertainty on $\alpha_s$ with
\begin{equation}
    \Delta^{\alpha_s}\sigma = \frac{\sigma(\alpha_s=0.1195)-\sigma(\alpha_s=0.1165)}{2}
\end{equation}

Since the correlation between $\alpha_s$ and the \acp{pdf} is small their uncertainties are applied quadratically combined \citep{unc_recipe,Butterworth_2016}.

\begin{equation}
    \Delta^{\text{PDF}+\alpha_s}\sigma=\sqrt{(\Delta^\text{PDF}\sigma )^2+(\Delta^{\alpha_s}\sigma)^2}
\end{equation}

\subsection{Uncertainty on HH cross section}
The cross-section uncertainties for the \ac{vbf} Higgs pair production process from upper mentioned approaches are for the scale variations \red{$^{+0.03\%}_{-0.04\%}$ and the combined \ac{pdf}+$\alpha_s$ uncertainty is $\pm 2.1\%$ }\citep{de2016arxiv}.

\subsection{Parton Shower}
Uncertainties related to the parton showering are estimated using different modelings from \textsc{Pythia 8} and \textsc{Herwig 7}. The largest deviations from the nominal are used as uncertainties on the Higgs pair process. \red{TODO}


\subsection{Branching Ratio Uncertainty}
The error estimate for the branching ratio takes into account theoretical uncertainties (THU) and parametric uncertainties (PU) that are included in the \ac{sm} calculations. The theoretical uncertainties mainly considers missing higher orders while for the parameters $p$ the four leading non-negligible contributions of the strong coupling and the quark masses $p=\{\alpha_s,m_c,m_b,m_t\}$ are considered.

Parametric uncertainties are Gaussian errors and are added in quadrature which ensures unity in the Branching Ratio calculation \citep{de2016arxiv}. Theoretical uncertainties in turn are not Gaussian and would lead to underestimated errors and are therefore added linearly \citep{de2016arxiv}. By assuming a Higgs mass of \qty[]{125}{GeV} and considering that there are two Higgs decaying to two $b$-quarks the error on the branching ratio is
\begin{equation}
    \Delta\text{BR} = 2 \times \left(\Delta\text{BR}(\text{THU}) + \sqrt{\sum\nolimits_{p} \Delta\text{BR}(\text{PU}_{p})^2 }\right) = _{-3.5\%}^{+3.4\%}.
\end{equation}

\section{Statistical Uncertainties}
As discussed in the chapter \ref{sec:statistics} on statistics the bin content for histograms in this work follows a Poisson distribution. Therefore the standard error for $N$ events is the square root of the Poisson variance $\sigma=\sqrt{\text{Var}}=\sqrt{N}$. Since histograms are filled weighted $\sum_i w_i N_i$ this needs to be taken into account. By making use of the additive property and invariance with respect to constants of the variance a bin filled with weights $w_i$ can be written as
\begin{align}
    \sigma_\text{stat}^2 & = \text{Var}_\text{bin}\left(\sum_i w_i\right)
    =
    \underbrace{\sum_i \text{Var}(w_i \times 1\text{ event})}_{\text{Var}(i+j)=\text{Var}(i)+\text{Var}(j)}
    =
    \underbrace{\sum_i w_i^2\text{Var}(1\text{ event})}_{\text{Var}(aX)=a^2\text{Var}(X)} \\ \nonumber
                         & =\sum_i w_i^2\sqrt{(1\text{ event})},
\end{align}
so that the statistical error reads
\begin{equation}
    \sigma_\text{stat}^\text{bin}=\sqrt{\sum_i w_i^2}.
\end{equation}

\section{Background Derivation Uncertainties}
The \ac{qcd} background is estimated with the ABCD method from the control region as detailed in section \ref{sec:abcd}. The uncertainties are assessed through error propagation of the statistical uncertainties on the quantities used to calculate the weight.
\begin{equation}
    \Delta_\text{stat} w_\text{CR} = w_\text{CR} \sqrt{
        \left(\frac{\Delta_\text{stat} N_\text{CR}^\text{2Xbb}}{N_\text{CR}^\text{2Xbb}}\right)^2
        +
        \left(\frac{\Delta_\text{stat} N_\text{CR}^\text{1Xbb}}{N_\text{CR}^\text{1Xbb}}\right)^2
    }
    = \red{\qty[]{0.0}{\percent}}
\end{equation}
To estimate uncertainty of the background estimate in the \ac{sr} for the $i$-th bin, statistical uncertainties are propagated and estimated from the \ac{vr}
\begin{equation}
    \Delta N_\text{SR, i}^\text{2Xbb}
    =
    \sqrt{
        \left(\frac{\Delta_\text{stat} w_\text{CR}}{w_\text{CR}}\right)^2
        +
        \left(\frac{\Delta_\text{stat} N_\text{VR, i}^\text{1Xbb}}{N_\text{VR, i}^\text{1Xbb}}\right)^2
    }
    \approx \red{\qty[]{0.0}{\percent}}.
\end{equation}


\red{will say in results part that the bkg estimate is validated as it is within the statistical error estimate}
