\chapter{Introduction}
Understanding nature through first principles has always been a fundamental human endeavor. The \ac{sm} of particle physics currently stands as the most precise theory, articulating elementary particles and their interactions through symmetry principles. The discovery of the Higgs boson by the \ac{cms} \citep{higgs-cms} and \ac{atlas} \citep{higgs} collaborations at the \ac{lhc} in 2012, half a century after its theoretical prediction \citep{PhysRevLett.13.321,PhysRevLett.13.508}, is not just a validation of the scientific method but also fills a pivotal gap in our understanding of the universe's basic structure.

However, while this achievement validated a crucial aspect of the \ac{sm} it also underscores the  theory's limitations. As a \ac{qft}, parameters in the \ac{sm} like the mass of the Higgs boson  are subject to loop corrections that can lead to large contributions. For the observed Higgs mass these loop corrections must be extremely precise suggesting that our current understanding may represent only an effective theory \citep{peskin2016trail}. Additionally, the potential of the Higgs boson, instrumental in maintaining the \acp{sm} consistency, relies on the simplest conceivable form. The true complexity of this potential, the possibility of multiple Higgs bosons, or even the Higgs' status as a fundamental particle remain open questions with any deviation from \ac{sm} predictions signalling a sign of new physics \citep{PhysRevD.101.075023}.

These considerations extend to the stability of the universe itself, with current precision of measurements suggesting a meta-stable state \citep{Buttazzo:2013uya,devoto2022false}. This implies that our universe might be on the edge of a phase transition, potentially transitioning to a true vacuum state via quantum tunneling over cosmological timescales. Such a transition could replace known forces, particles and structures of the universe with different ones. This is intricately linked to the shape of the Higgs potential and requires more precise measurements to make accurate statements about such a phase transition. The discovery of the Higgs boson is therefore more than just a milestone; it acts as a portal to unexplored areas in physics \citep{dawsona2022report}.

The \ac{sm} encompasses 26 parameters, 15 of which are associated to the Higgs mechanism, emphasizing its centrality in the model \citep{thomson2013modern}. One prediction by the \ac{sm} is the process of Higgs boson pair production, a phenomenon under scrutiny by both the \ac{atlas} and \ac{cms} collaborations across various decay channels \citep{GOUZEVITCH2020100039}. This thesis concentrates on the search for Higgs boson pairs in a boosted topology, decaying into the four $b$-quark final state using the \ac{atlas} detector \citep{atlas2018search}.

The advent of machine learning in particle physics introduced powerful tools for classification problems but also challenges in optimizing these tools for the field's unique goals. These tools, often tailored for specific tasks, may not align with the overarching goals of discovering new particles or verifying new theories, leading to suboptimal outcomes \citep{guest2018deep}. This gap primarily arises from depending on intermediate optimization metrics that disregard systematic uncertainties' significant role in test statistical tests used for theory confirmation. The innovative \textsc{neos} approach \citep{Simpson_2023} introduces a solution to this problem, for which this work demonstrates a first application of a systematic-aware optimization for sensitivity in a particle physics experiment.

This thesis is structured to first introduce the \ac{sm} and the \ac{atlas} detector followed by a comprehensive discussion of analysis methods, including the reconstruction of physical objects, the analysis strategy and event selection, machine learning for systematic-aware neural network training, systematic uncertainties and the framework used for the evaluation of statistical tests. \red{The results section presents a strategy for improved $b$-quark identification using muons and findings on Higgs pair production cross-section limits using Run 2 data.}



% This thesis is structured to first introduce the \ac{sm} \ref{ch:sm} and the \ac{atlas} detector \ref{sec:atlas}, followed by a comprehensive discussion of analytical methodologies, including the reconstruction of physical objects \ref{ch:reco}, analysis strategy and event selection \ref{ch:hh4b}, machine learning for systematic-aware neural network training \ref{sec:analysis_optimization}, systematic uncertainties \ref{ch:systematics} and the framework used for the evaluation of statistical tests \ref{sec:statistics}. The results section presents a strategy for improved $b$-quark identification using muons \ref{ch:smt} and findings on Higgs pair production cross-section limits using Run 2 data \ref{ch:hh4b-results}.



% latest limits: https://arxiv.org/pdf/2301.03212.pdf 
% hh review 
% https://www.sciencedirect.com/science/article/pii/S2405428320300022#fig0002


% The current best limits 
% atlas 2021 
% of 𝜅2𝑉 Ÿ 0•43 and 𝜅2𝑉 ¡ 2•56
% https://arxiv.org/pdf/2001.05178.pdf 

% depending on if current r21 comes out before 

% Prospects of non-resonant Higgs pair production
% at the HL-LHC and HE-LHC
% https://iopscience.iop.org/article/10.1088/1742-6596/1690/1/012149/pdf


% https://indico.cern.ch/event/1359386/contributions/5723345/attachments/2786832/4859053/ATLASdihiggs2024jan.pdf 

