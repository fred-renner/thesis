\chapter{Introduction}
% Understanding nature from its foundational principles has always been at the heart of human curiosity. 
Understanding nature through first principles is an intrinsic human endeavor. The \ac{sm} of particle physics currently stands as the most precise theory, articulating elementary particles and their interactions through symmetry principles. The landmark discovery of the Higgs boson by the \ac{cms} \citep{higgs-cms} and \ac{atlas} \citep{higgs} collaborations at the \ac{lhc} in 2012, a half-century after its theoretical prediction \citep{PhysRevLett.13.321,PhysRevLett.13.508}, stands not only as a testament of the scientific method but also filled a pivotal gap in our understanding of the universe's fundamental structure.

Yet, this achievement opens the door to new questions regarding the \acp{sm} completeness and consistency. As a \ac{qft}, the \ac{sm} is subject to loop corrections that demand extreme precision for the observed Higgs mass \citep{peskin2016trail}, implying that our current understanding may represent only an effective theory, hinting at the existence of undiscovered physics. Additionally, the conventional potential of the Higgs boson, instrumental in maintaining the \acp{sm} consistency, relies on the simplest conceivable form. The true complexity of this potential, the possibility of multiple Higgs bosons, or even the Higgs' status as a fundamental particle remain open questions with any deviation from \ac{sm} predictions signalling a sign of new physics \citep{PhysRevD.101.075023}. These considerations extend to the stability of the universe itself, with current precision of measurements suggesting a meta-stable state \citep{Buttazzo:2013uya}. The discovery of the Higgs boson is therefore more than just a milestone; it acts as a portal to uncharted territories in physics and thus requires more precise measurements \citep{dawsona2022report}.

The \ac{sm} encompasses 26 parameters, 15 of which are determined by the Higgs mechanism, emphasizing its centrality in the model \citep{thomson2013modern}. One prediction is the process of Higgs boson pair production, a phenomenon under rigorous scrutiny by both the \ac{atlas} and \ac{cms} collaborations across various decay channels \citep{GOUZEVITCH2020100039}. This thesis concentrates on the search for Higgs boson pairs in a boosted topology, decaying into the four $b$-quark final state using the \ac{atlas} detector \citep{atlas2018search}.

The advent of machine learning in particle physics introduced powerful tools for classification problems but also challenges in optimizing these tools for the field's unique goals. These tools, often tailored for specific tasks, may not align with the overarching goals of discovering new particles or verifying new theories, leading to suboptimal outcomes \citep{guest2018deep}.
This gap primarily arises from depending on intermediate optimization metrics that disregard systematic uncertainties' significant role in the statistical test's validity for theory confirmation. The innovative \textsc{neos} approach \citep{Simpson_2023} introduces a solution to this problem, for which this work demonstrates a first application of of a systematic-aware optimization on sensitivity in a particle physics experiment.

This thesis is structured to first introduce the \ac{sm} \ref{ch:sm} and the \ac{atlas} detector \ref{sec:atlas}, followed by a comprehensive discussion of analytical methodologies, including the reconstruction of physical objects \ref{ch:reco}, analysis strategy and event selection \ref{ch:hh4b}, machine learning for systematic-aware neural network training \ref{sec:analysis_optimization}, systematic uncertainties \ref{ch:systematics} and the framework used for the evaluation of statistical tests \ref{sec:statistics}. The results section presents a strategy for improved $b$-quark identification using muons \ref{ch:smt} and findings on Higgs pair production cross-section limits using Run 2 data. \ref{ch:hh4b-results}.





% latest limits: https://arxiv.org/pdf/2301.03212.pdf 
% hh review 
% https://www.sciencedirect.com/science/article/pii/S2405428320300022#fig0002


% The current best limits 
% atlas 2021 
% of 𝜅2𝑉 Ÿ 0•43 and 𝜅2𝑉 ¡ 2•56
% https://arxiv.org/pdf/2001.05178.pdf 

% depending on if current r21 comes out before 

% Prospects of non-resonant Higgs pair production
% at the HL-LHC and HE-LHC
% https://iopscience.iop.org/article/10.1088/1742-6596/1690/1/012149/pdf


% https://indico.cern.ch/event/1359386/contributions/5723345/attachments/2786832/4859053/ATLASdihiggs2024jan.pdf 

