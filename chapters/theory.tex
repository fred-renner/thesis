\chapter{Theory}

Where to start? Sometimes it semms like that particle physics is bringing it all together, as it tries to give a comprehensive picture of the world by describing the structure of matter from quantum mechanics to cosmology. So I would say to start shallow (since we are experimentalists) at the very beginning, and then dive a bit deeper into Standard Model to have a plausible thread how the need and the development of the Higgs mechanism came about. The first section is based on \citet{zee2010quantum} and intended to make the calculation of cross sections plausible.

\section{The field theory idea for the experimental particle physicist}
The fact that elementary particles can seemingly be born out of nothing and die again led to the development of their currently most successful description through quantum field theories. Heuristically it can be understood by the uncertainty principle, which states that energy can vary greatly on short time scales, and by special relativity, which allows the property energy to be converted into the property mass. This marriage between quantum mechanics and special relativity is what drove the development of quantum field theory. 

To make a field one equips the room with a quantity that depends on the position so e.g. $\phi(\bm{x},t)$. A Lagrangian $L(\phi(\bm{x},t))$ then governs the dynamics, like excitations or interactions of this field, which can represented for example the birth and death of particles or interactions by the exchange of a particle between them. By use of the path integral formulation it basically boils down to integrals of the form $\int D\phi e^{i\int d^4x L(\phi(\bm{x},t))}$. Where $\int D\phi$ is the sum over all possible paths/ways a particle could take. Through back and forth expansions of the $e$ functions the integral can be solved and the result is a probability - the amplitude of e.g. an interaction between two particles, like scattering, usually depicted in the form of Feynman Diagrams. As this follows a pattern the formalism can be contracted into the infamous Feynman rules. 

Probing elementary particle interactions are accessible via bound states, decays and scattering. The first can be studied within classical quantum mechanics whereas the latter uses the preceding. Since this work deals with a collider experiment the tool at hand is the cross section $\sigma$. It is a measure of how possible an interaction is when shooting something at each other. Calculating reaction rates in quantum mechanics is done by Fermi's golden rule. Here the relativistic version for a scattering process like $1+2 \rightarrow 3+4+\dots+n$ is given (see \citep{griffiths2020introduction})
\begin{equation}
    \sigma=\frac{S\hbar^2}{4\sqrt{(p_1\dot p_2)^2 - (m_1m_2c^2)^2}}\int \abs{\mathcal{M}}^2(2\pi)^4\delta^4(p_1+p_2-p_3 \dots-p_n) 
    \times \prod_{j=3}^n 2\pi \delta(p_j^2-m_j^2c^2)\Theta(p_j^0))\frac{\mathrm
    d^4p_j}{(2\pi)^4}.
\end{equation}




To probe nature at the smallest scales 

As we can produce e.g. in a collider experiment 

Equipped with this on can calculate cross sections which is nothing 


To get an insight at the smallest scales all we can apparently do is to 

xsec berechnen
warum xsec 
wollen wissen wie wahrscheinlich in process ist 


This is the low level idea how particle physicists calculate probabilities or amplitudes for a scattering process of interest. 


dirac, require local gauge invariance -> qed Lagrangian
